\documentclass[10pt,a4paper]{article}
\usepackage{fontspec}
\usepackage[slantfont, boldfont]{xeCJK}


\XeTeXlinebreaklocale "zh"
\XeTeXlinebreakskip = 0pt plus 1pt minus 0.1pt

\setCJKmainfont{Adobe Heiti Std}
%\setCJKmainfont{Adobe Song Std}
\parindent 2em

\begin{document}
\title{图形学 篮球-第二阶段}
\author{宋文灏 5090379015}
\date{\today}
\maketitle
\section{建模}
\subsection{球场看台}
考虑到程序的效率和实现的简便,球场四周的看台采用长方形平面+纹理映射的方式来显示。纹理的原图是通过网络搜索获取,然后再对图片进行剪裁和转换格式,作为纹理使用。
\subsection{篮球架和篮框}
参考github上看到一个篮球建模的例子,https://github.com/raananw/BasketBall-trainer-openGL,用基本图形拼出篮球架。
\subsection{篮球}
与第一阶段相同,用opengl画一个球体。

\section{聚光灯效果}
聚光灯本身并不是很难实现。相比于普通的全局光或是点光源,用opengl里的glLightfv设一下各种参数就行了。

在做聚光灯效果的过程中,我遇到过两个问题。一个是我一直把GL\_SPO\\T\_DIRECTION错认为是目标的点坐标而不是方向向量,导致光照不到地板上。这个问题调了整整一个上午,直到一位同学跟我指出这个问题,我才最终让地板上出现了光亮。另一个是聚光灯光照在地板上只有一个渐变的效果而没有聚光灯效果。后经另一位同学提醒,找到了原因---地板用四个顶点绘制太少了,于是把地板分成许多个小的长方体进行绘制,这个问题就解决了。但是这样绘制出来,聚光灯的边缘比较毛糙。后来看网上别的样例代码的时候,发现有人用三角形绘制图形,忽然发现是不是用三角形来绘制地板会有更好的效果。只是想了想,有些怕麻烦,就没有去试。

\section{纹理映射}
\subsection{地板}
纹理的选择和第一阶段的差不多一样。不同的是,因为地板分成许多小的长方体来画,glTexCoord2f就对应纹理的一小部分。
\subsection{球场看台}
从网上找了一张NBA 2K的游戏截图,进行了一下剪裁和格式转化,就用作了看台的纹理,远看的话效果还不错。
\subsection{篮板}
用的是BasketBall-trainer-openGL里的贴图。作者把贴图的数据以字符串的形式写在了头文件里。我就把他的代码直接拿来用了。

\section{篮球阴影}
参考了opengl superbible第四章阴影的例子,通过矩阵运算把篮球投影到地板平面上。上一个阶段用的也是这个,阴影有一点问题,这次研究了一下,把阴影的位移变化放在了矩阵变换后,这样虽然比上次更“真”了一点,但阴影的位置还是有问题。我也尝试过opengl superbible第十四章的shadowmap的方法,通过光视角中的场景放进纹理,再对地板进行绘制。无奈这个个不仅没有产生阴影,连地板的纹理也出了问题,只好回到了之前的矩阵投影方法。

\section{编译和运行}
编译程序,首先,你需要安装qt。

这个项目使用qt的qmake来进行项目的编译管理。在终端里里cd到项目的文件夹下,qmake,make,就可以完成编译了。

我是在linux下开发的,linux下编译需要安装gl,glu,glut的库。

windows里需要的库都已经包含在了项目文件夹里,可以直接编译。但是运行编译出来的可执行文件似乎会卡住。可能是windows下opengl库或mingw库的的问题。

不妨先试一下运行可执行程序,如果在你的电脑上也不能正常运行,我另外附上了该程序在linux下运行的屏幕录像。如果需要现场的演示,可以联系我,邮箱wenhao.sng@gmail.com, 手机13817724887。演示的时间最好是在周一至周五晚上六点后,或是双休日。

\section{程序操作}
\begin{itemize}
  \item q w e a s d 进行视角的移动
  \item i j k l o u 控制篮球的移动
  \item b 开关环境灯
  \item r 重置篮球的位置和视角
\end{itemize}

\end{document}
